% Faz com que o ínicio do capítulo sempre seja uma página ímpar
%\cleardoublepage
% Inclui o cabeçalho definido no meta.tex
\pagestyle{fancy}
% Números das páginas em arábicos
\pagenumbering{arabic}

\chapter{Introdução}\label{intro}

A história tem experimentado mais uma transformação de paradigma  sócio-econômico de proporções planetárias com o aumento do acesso à internet. Temos passado da era Industrial, que foi precedida pela sociedade agrária, para novos momentos de ruptura, em a produção e o compartilhamento de informações através da participação em redes sociais pode ser notada como um meio de produção disruptivo e princípios como colaboração, transparência, compartilhamento e empoderamento conduzem as pessoas a um novo patamar de liberdade, consciência e inovação (TAPSCOTT; WILLIAMS, 2007).
Toda essa mudança de direcionamento têm sido possível em grande parte graças à produção colaborativa baseada em recursos comuns ー termo cunhado por Benkler (2002) ー que vai ao encontro da reflexão de Lévy (2007), [sobre  uma] inteligência distribuída por toda parte, incessantemente valorizada, coordenada em tempo real, que resulta na mobilização efetiva das competências.
Benkler e Nissenbaum (2006) explicam que a produção colaborativa baseada em recursos comuns nada mais é do que “uma forma de produção sócio-econômica emergente das redes sociais, facilitada pela infraestrutura da internet e que se expressa através de um sistema sócio-técnico caracterizado principalmente pela colaboração entre grupos massivos de indivíduos para a promoção de ambientes informacionalmente ricos, sem dependência nem da remuneração praticada pelo mercado de trabalho nem da existência de hierarquia para a coordenação desse empreendimento comum, a que Benkler, 2006 chama “Commons”. 
O Commons, na visão de Benkler, 2017 é uma família de mecanismos institucionais que evita o controle exclusivo sobre a propriedade e utiliza uma gama diversificada de mecanismos de governança social para gerenciar a utilização de recursos e projetos que não sejam propriedade e contrato.
Exemplos notáveis e históricos de produção colaborativa baseada em recursos comuns estão, segundo Benkler e Nissenbaum (2006), no desenvolvimento de software livres como o sistema operacional GNU/Linux, o servidor web Apache e a linguagem de programação multiplataforma Pearl. Exemplos mais contemporâneos de produção de bens comuns por pares em rede, podem ser exemplificados pelas plataformas Kaggle, para Ciência de dados, Quora, para quaisquer tipos de perguntas e Duolingo, para a aprendizagem de idiomas, todos possuindo comunidades atuantes e diferentes níveis de governança. 
Tapscott e Williams (2007) argumentam que a utilização da inteligência coletiva voluntária para a resolução de problemas funciona porque “a nova economia desencadeada pela tecnologia alterou permanentemente os custos e benefícios de produzir informações e colaborar”. A essa mudança de paradigma, Don Tapscott e Anthony D. Williams chamaram “Wikinomics”, criando um neologismo em alusão ao impacto no modo de produção provocado pela vida em rede (TAPSCOTT; WILLIAMS, 2007). 
É Interessante notar que a disponibilização de conteúdo/informação esteve na vanguarda dos processos de mudança sócio-econômica, desde a invenção da escrita, que marca a separação entre história e pré-história, sendo também um dos pilares importantes da transição da sociedade agrária para a sociedade industrial com surgimento da imprensa de Gutenberg. 
Como um alicerce à uma nova transformação de proporções planetárias,  a Internet, através da Web 2.0 e seu potencial de promoção de colaboração entre humanos facilitou intercâmbios de conteúdos através dos diversos formatos de mídia, contando com a enorme diversidade oportunizada pela comunicação em rede, ao que Surowiecki (2006) chamou Sabedoria das Multidões. Esse conceito, traduzido da expressão “Wisdom of Crowd” caracteriza-se, segundo Surowiecki (2006) por um ambiente que satisfaz determinadas condições para que um grupo seja inteligente e, portanto, mais habilidoso na tomada de decisões. 
E essas condições, de acordo com Surowiecki (2006) são: diversidade de opinião (cada pessoa deve ter alguma opinião própria), mesmo que seja apenas uma informação excêntrica dos fatos conhecidos), independência (as opiniões das pessoas não são determinadas pelas opiniões daqueles que as cercam), descentralização (as pessoas são capazes de se especializar e trabalhar com conhecimento local) e agregação (a existência de algum mecanismo para transformar avaliações pessoais em uma decisão coletiva). 
Para que o fenômeno da sabedoria das multidões seja observado, é necessário a proposição de um ambiente crowdsourcing. Crowdsourcing, segundo o Oxford Living Dictionaries é a prática de obter informações ou entrada em uma tarefa ou projeto, alistando os serviços de um grande número de pessoas, pagas ou não, normalmente através da Internet. 
OOMEN e AROYO (2011) apresentam uma classificação simplificada das Iniciativas Crowdsourcing, apresentando como tipos: Correção e Tarefas de Transcrição de saídas de processos de digitalização; Complementação de Coleções a serem exibidas na web; Classificação (Reunião de metadados descritivos relacionados a objetos em uma coleção); Co-curadoria para criação de exposições web; Financiamento Colaborativo e Contextualização (adição de conhecimento contextual a objetos), como no caso da escrita de páginas wiki. 
O termo Wiki foi criado por Ward Cunningham, programador norte-americano que criou o primeiro wiki em 1995. Wiki tem origem lexicográfica no termo havaiano “wiki-wiki-quick” que significa “rapidinho”, aprendido por Cunningham durante uma viagem. Raman (2006) alerta para o fato de que a palavra Wiki (grafada com letra inicial maiúscula) se refere à tecnologia, enquanto wiki se refere à aplicação em si. 
O Wiki é uma coleção livremente expansível de páginas interligadas com o objetivo de facilitar o trabalho cooperativo e a geração de conhecimento, funcionando como um sistema de gerenciamento de conteúdo, um fórum de discussão e outras formas de groupware (RAMAN, 2006). 
Ainda segundo Raman (2006), o Wiki funciona através de um sistema de hipertextos, com potencial para a guarda e a modificação da informação ー uma base de dados onde cada página é facilmente editável por qualquer usuário.
A Wikipedia é a maior enciclopédia digital exemplo de Wiki, que tem ampliado sobremaneira a quantidade de registros, a diversidade, a penetração sócio-espacial e a velocidade de compartilhamento do estado atual do conhecimento humano, consequência da democratização dos acesso e produção de informação decorrentes da estratégia/modelo crowdsourcing de negócio adotada pelos seus criadores, Jimmy Wales e Larry Sanger, que anteriormente haviam criado a Nupedia, que não teve sucesso, devido ao seu modelo de negócios mais restritivo com respeito à colaboração.
Antes da Wikipedia, o conhecimento era encontrado em compilações como a Enciclopédia Britânica, entre outras. A Wikipedia surgiu no ano 2001 e a motivação para a sua criação foi a oportunidade de desenvolver uma enciclopédia online gratuita que qualquer um pode editar (JIM GILES, 2005). 
Em julho de 2018, a cobertura da Wikipedia conta com aproximadamente 48.285.498 verbetes, que são dispostos cada um em uma página exclusiva. Considerando as informações que constam na página principal do projeto foram editados mais de: 5.697.688 verbetes em Inglês, 5.359.690 em Cebuano, 3.766.268 em Sueco, 2.221.132 em Alemão, 1.979.705 em Francês,  1.931.419 em Holandês 1.496.850 em Russo, 1.448.564 em Italiano, 1.448.564 em Espanhol e 1.303.708 artigos em Polonês e 21.631.903 em ainda outras línguas, sendo cerca de 290 línguas diferentes.

Gráfico 1: Cobertura da Wikipedia segundo cada vernáculo empregado. O Inglês destaca-se como língua mais presente. A Wikipedia lusófona participa com mais de 1.000.000. menos de 2,1% do número total de artigos.

