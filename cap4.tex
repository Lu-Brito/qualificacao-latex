%\pagestyle{empty}
%\cleardoublepage
\pagestyle{fancy}

\chapter{Metodologia}\label{cap4}

Metodologia aqui!

% Outro ambiente útil é o description, como exemplificado abaixo.
\begin{description}
  \item[Estágio1 $(n=27)$:] Descrição minuciosa deste estágio.
    Estou incluindo um pouco de texto extra para mostrar como a formatação fica impecável.
    Uma boa formatação não distrai o leitor e proporciona maior clareza e prazer durante a leitura.
  \item[Estágio2 $(n=25)$:] Descrição minuciosa deste estágio.
    Estou incluindo um pouco de texto extra para mostrar como a formatação fica impecável.
    Uma boa formatação não distrai o leitor e proporciona maior clareza e prazer durante a leitura.
\end{description}

As descrições também podem ser colocadas uma dentro da outra.

\begin{description}
  \item[Tipo1:] Descrição minuciosa.
    Estou incluindo um pouco de texto extra para mostrar como a formatação fica impecável.
    A razão $\frac{\text{núcleo}}{\text{citoplasma}}\times100=51,0\pm\unit[11,9]{\%}$.
  \item[Tipo2:] ~
    \begin{description}
      \item[Subtipo2.1:] Descrição minuciosa deste tipo.
	Estou incluindo um pouco de texto extra para mostrar como a formatação fica impecável.
      \item[Subtipo2.2:] Descrição minuciosa deste tipo.
	Estou incluindo um pouco de texto extra para mostrar como a formatação fica impecável.
    \end{description}
  \item[Tipo3:] Descrição minuciosa deste tipo.
	Estou incluindo um pouco de texto extra para mostrar como a formatação fica impecável.
\end{description}
